\documentclass[]{article}
\usepackage{lmodern}
\usepackage{amssymb,amsmath}
\usepackage{ifxetex,ifluatex}
\usepackage{fixltx2e} % provides \textsubscript
\ifnum 0\ifxetex 1\fi\ifluatex 1\fi=0 % if pdftex
  \usepackage[T1]{fontenc}
  \usepackage[utf8]{inputenc}
\else % if luatex or xelatex
  \ifxetex
    \usepackage{mathspec}
  \else
    \usepackage{fontspec}
  \fi
  \defaultfontfeatures{Ligatures=TeX,Scale=MatchLowercase}
\fi
% use upquote if available, for straight quotes in verbatim environments
\IfFileExists{upquote.sty}{\usepackage{upquote}}{}
% use microtype if available
\IfFileExists{microtype.sty}{%
\usepackage{microtype}
\UseMicrotypeSet[protrusion]{basicmath} % disable protrusion for tt fonts
}{}
\usepackage[margin=1in]{geometry}
\usepackage{hyperref}
\hypersetup{unicode=true,
            pdftitle={Lesson 02 - Getting your System Setup},
            pdfborder={0 0 0},
            breaklinks=true}
\urlstyle{same}  % don't use monospace font for urls
\usepackage{color}
\usepackage{fancyvrb}
\newcommand{\VerbBar}{|}
\newcommand{\VERB}{\Verb[commandchars=\\\{\}]}
\DefineVerbatimEnvironment{Highlighting}{Verbatim}{commandchars=\\\{\}}
% Add ',fontsize=\small' for more characters per line
\usepackage{framed}
\definecolor{shadecolor}{RGB}{248,248,248}
\newenvironment{Shaded}{\begin{snugshade}}{\end{snugshade}}
\newcommand{\AlertTok}[1]{\textcolor[rgb]{0.94,0.16,0.16}{#1}}
\newcommand{\AnnotationTok}[1]{\textcolor[rgb]{0.56,0.35,0.01}{\textbf{\textit{#1}}}}
\newcommand{\AttributeTok}[1]{\textcolor[rgb]{0.77,0.63,0.00}{#1}}
\newcommand{\BaseNTok}[1]{\textcolor[rgb]{0.00,0.00,0.81}{#1}}
\newcommand{\BuiltInTok}[1]{#1}
\newcommand{\CharTok}[1]{\textcolor[rgb]{0.31,0.60,0.02}{#1}}
\newcommand{\CommentTok}[1]{\textcolor[rgb]{0.56,0.35,0.01}{\textit{#1}}}
\newcommand{\CommentVarTok}[1]{\textcolor[rgb]{0.56,0.35,0.01}{\textbf{\textit{#1}}}}
\newcommand{\ConstantTok}[1]{\textcolor[rgb]{0.00,0.00,0.00}{#1}}
\newcommand{\ControlFlowTok}[1]{\textcolor[rgb]{0.13,0.29,0.53}{\textbf{#1}}}
\newcommand{\DataTypeTok}[1]{\textcolor[rgb]{0.13,0.29,0.53}{#1}}
\newcommand{\DecValTok}[1]{\textcolor[rgb]{0.00,0.00,0.81}{#1}}
\newcommand{\DocumentationTok}[1]{\textcolor[rgb]{0.56,0.35,0.01}{\textbf{\textit{#1}}}}
\newcommand{\ErrorTok}[1]{\textcolor[rgb]{0.64,0.00,0.00}{\textbf{#1}}}
\newcommand{\ExtensionTok}[1]{#1}
\newcommand{\FloatTok}[1]{\textcolor[rgb]{0.00,0.00,0.81}{#1}}
\newcommand{\FunctionTok}[1]{\textcolor[rgb]{0.00,0.00,0.00}{#1}}
\newcommand{\ImportTok}[1]{#1}
\newcommand{\InformationTok}[1]{\textcolor[rgb]{0.56,0.35,0.01}{\textbf{\textit{#1}}}}
\newcommand{\KeywordTok}[1]{\textcolor[rgb]{0.13,0.29,0.53}{\textbf{#1}}}
\newcommand{\NormalTok}[1]{#1}
\newcommand{\OperatorTok}[1]{\textcolor[rgb]{0.81,0.36,0.00}{\textbf{#1}}}
\newcommand{\OtherTok}[1]{\textcolor[rgb]{0.56,0.35,0.01}{#1}}
\newcommand{\PreprocessorTok}[1]{\textcolor[rgb]{0.56,0.35,0.01}{\textit{#1}}}
\newcommand{\RegionMarkerTok}[1]{#1}
\newcommand{\SpecialCharTok}[1]{\textcolor[rgb]{0.00,0.00,0.00}{#1}}
\newcommand{\SpecialStringTok}[1]{\textcolor[rgb]{0.31,0.60,0.02}{#1}}
\newcommand{\StringTok}[1]{\textcolor[rgb]{0.31,0.60,0.02}{#1}}
\newcommand{\VariableTok}[1]{\textcolor[rgb]{0.00,0.00,0.00}{#1}}
\newcommand{\VerbatimStringTok}[1]{\textcolor[rgb]{0.31,0.60,0.02}{#1}}
\newcommand{\WarningTok}[1]{\textcolor[rgb]{0.56,0.35,0.01}{\textbf{\textit{#1}}}}
\usepackage{graphicx,grffile}
\makeatletter
\def\maxwidth{\ifdim\Gin@nat@width>\linewidth\linewidth\else\Gin@nat@width\fi}
\def\maxheight{\ifdim\Gin@nat@height>\textheight\textheight\else\Gin@nat@height\fi}
\makeatother
% Scale images if necessary, so that they will not overflow the page
% margins by default, and it is still possible to overwrite the defaults
% using explicit options in \includegraphics[width, height, ...]{}
\setkeys{Gin}{width=\maxwidth,height=\maxheight,keepaspectratio}
\IfFileExists{parskip.sty}{%
\usepackage{parskip}
}{% else
\setlength{\parindent}{0pt}
\setlength{\parskip}{6pt plus 2pt minus 1pt}
}
\setlength{\emergencystretch}{3em}  % prevent overfull lines
\providecommand{\tightlist}{%
  \setlength{\itemsep}{0pt}\setlength{\parskip}{0pt}}
\setcounter{secnumdepth}{0}
% Redefines (sub)paragraphs to behave more like sections
\ifx\paragraph\undefined\else
\let\oldparagraph\paragraph
\renewcommand{\paragraph}[1]{\oldparagraph{#1}\mbox{}}
\fi
\ifx\subparagraph\undefined\else
\let\oldsubparagraph\subparagraph
\renewcommand{\subparagraph}[1]{\oldsubparagraph{#1}\mbox{}}
\fi

%%% Use protect on footnotes to avoid problems with footnotes in titles
\let\rmarkdownfootnote\footnote%
\def\footnote{\protect\rmarkdownfootnote}

%%% Change title format to be more compact
\usepackage{titling}

% Create subtitle command for use in maketitle
\providecommand{\subtitle}[1]{
  \posttitle{
    \begin{center}\large#1\end{center}
    }
}

\setlength{\droptitle}{-2em}

  \title{Lesson 02 - Getting your System Setup}
    \pretitle{\vspace{\droptitle}\centering\huge}
  \posttitle{\par}
    \author{}
    \preauthor{}\postauthor{}
      \predate{\centering\large\emph}
  \postdate{\par}
    \date{Last Updated 06-14-2019}


\begin{document}
\maketitle

This lesson will show you how to get your system setup to work in the
modern data analysis environment R Studio.

\begin{quote}
If you are using an iPad or Chromebook or otherwise do not have a
computer that you can install programs on, head over to
\url{http://rstudio.cloud.com}, make an account and start a new project.
Then go watch the video below on how to navigate R studio.
\end{quote}

While the cloud is easier to initially setup, I want you to only use it
if you absolutely have to. Having your own installation on your computer
ensures that

\begin{itemize}
\tightlist
\item
  you likely want to customize your programs
\item
  you will be able to put your files under version control
\item
  you always have access to your code even with unstable or no internet
\item
  Honestly, it may not always be free. This is an alpha testing stage
  for the company.
\end{itemize}

R is the programming language that we will be using. R Studio is the
program we will \emph{use} R through. You need to install both.

\hypertarget{download-and-install-r}{%
\section{1. Download and install R}\label{download-and-install-r}}

\begin{itemize}
\tightlist
\item
  Download R (v3.5+) from \url{https://cran.r-project.org/}

  \begin{itemize}
  \tightlist
  \item
    Direct link for Mac OS X El Capitan or higher:
    \url{https://cran.r-project.org/bin/macosx/R-3.6.0.pkg}
  \item
    Direct link for Windows 10
    \url{https://cran.r-project.org/bin/windows/base/R-3.6.0-win.exe}
  \end{itemize}
\item
  Install R by double clicking on the downloaded file and following the
  prompts. Default settings are OK.

  \begin{itemize}
  \tightlist
  \item
    Delete any desktop shortcut that was created (looks like the icon
    above.)
  \end{itemize}
\end{itemize}

\hypertarget{download-and-install-r-studio}{%
\section{2. Download and install R
Studio}\label{download-and-install-r-studio}}

\begin{itemize}
\tightlist
\item
  Download R Studio (v 1.2+) from
  \url{https://www.rstudio.com/products/rstudio/download/\#download}

  \begin{itemize}
  \tightlist
  \item
    Choose the download link that corresponds to your operating system.
  \end{itemize}
\item
  Install R Studio

  \begin{itemize}
  \tightlist
  \item
    Windows: Double click on the downloaded file to run the installer
    program.
  \item
    Mac: Double click on the downloaded file, then drag the R Studio
    Icon into your Applications folder.

    \begin{itemize}
    \tightlist
    \item
      After you are done, eject the ``Drive'' that you downloaded by
      dragging it to your trash.
    \end{itemize}
  \end{itemize}
\item
  Watch the following short video to learn how to navigate R Studio.
\end{itemize}

RStudio IDE Overview

\hypertarget{setting-preferences-in-r-studio-to-retain-sanity-while-debugging}{%
\section{3. Setting preferences in R Studio to retain sanity while
debugging}\label{setting-preferences-in-r-studio-to-retain-sanity-while-debugging}}

\begin{itemize}
\tightlist
\item
  Open R Studio and go to the file menu go to Tools then Global Options.
\item
  Uncheck ``Restore .RData into workspace at startup''
\item
  Where it says ``Save workspace to .RData on exit:'' Select ``Never''"
\item
  Click apply then ok to close that window.
\end{itemize}

This will ensure that when you restart R you do not ``carry forward''
objects such as data sets that you were working on in a prior
assignment.

You should make a habit to completely shut down R studio when you are
done working. Your open tabs will be saved, but your environment will be
cleared. \emph{This is a good thing.}

\hypertarget{set-up-a-class-folder}{%
\section{4. Set up a class folder}\label{set-up-a-class-folder}}

\begin{itemize}
\item
  Create a class folder in an easy to find place on your computer using
  one of the options below:

  \begin{itemize}
  \tightlist
  \item
    ALL CAPS (MATH130)
  \item
    no caps (math130)
  \item
    snake\_case (math\_130)
  \item
    CamelCase (Math130)
  \end{itemize}
\item
  In this \texttt{math130} folder, create a subfolder named
  \texttt{data}, and a second subfolder for \texttt{homework}.
\item
  You will put all files related to this class in here.

  \begin{itemize}
  \tightlist
  \item
    Lecture notes go in the main folder, homework in the
    \texttt{homework} folder, and data in, you guessed it, the
    \texttt{data} folder.
  \end{itemize}
\item
  This means when you download a file, right click and ``Save as'' or
  ``Save target as'' and \textbf{actively choose} where to download this
  file. Do not let files live in your downloads folder.
\item
  🚫 Do not open any files from your browser window after downloading.
\end{itemize}

Test this out now by clicking \href{hw/hw1.Rmd}{{[}this link{]}} to
download Homework 1 assignment now.

\hypertarget{installing-packages}{%
\section{5. Installing Packages}\label{installing-packages}}

R is considered an \textbf{Open Source} software program. That means
many (thousands) of people contribute to the software. They do this by
writing commands (called functions) to make a particular analysis
easier, or to make a graphic prettier.

When you download R, you get access to a lot of functions that we will
use. However these other \emph{user-written} packages add so much good
stuff that it really is the backbone of the customizability and
functionality that makes R so powerful of a language.

For example we will be creating graphics using functions like
\texttt{boxplot()} and \texttt{hist()} that exist in base R. But we will
quickly move on to creating graphics using functions contained in the
\texttt{ggplot2} package. We will be managing data using functions in
\texttt{dplyr} and reading in Excel files using \texttt{readxl}.
Installing packages will become your favorite pasttime.

To get started open R Studio and in the Console type the following
command to install the \texttt{rmarkdown} package. Hit enter to submit
the command to R and watch the install.

\begin{Shaded}
\begin{Highlighting}[]
\KeywordTok{install.packages}\NormalTok{(}\StringTok{"ggplot2"}\NormalTok{)}
\end{Highlighting}
\end{Shaded}

When the console returns to showing a \texttt{\textgreater{}} (and the
stop sign in the top right corner of the Console window goes away), R is
done doing what you asked it to do and ready for you to give it another
command. You should get a message that looks similar to the message
below.

\begin{verbatim}
The downloaded binary packages are in
    C:\Users\Robin\AppData\Local\Temp\Rtmpi8NAym\downloaded_packages
\end{verbatim}

Some packages use functions that live in other packages. These are
called dependencies and will be automatically installed when you tell R
to install a new package. 99\% of the time you will not have to install
them manually on your own.

\begin{quote}
You only have to install packages once. To access commands within the
functions you must load the package first using the code
\texttt{library(package\ name)}. We will start to use various packages
next week.
\end{quote}

Now that you're a package installing pro, go ahead and install the
following packages that we will be using in the next few weeks. Type
these commands into the console one at a time, hitting enter to submit,
and waiting for it to finish before entering the next command. \textbf{R
is case sensitive}.

\begin{Shaded}
\begin{Highlighting}[]
\KeywordTok{install.packages}\NormalTok{(}\StringTok{"rmarkdown"}\NormalTok{)}
\KeywordTok{install.packages}\NormalTok{(}\StringTok{"knitr"}\NormalTok{)}
\KeywordTok{install.packages}\NormalTok{(}\StringTok{"forcats"}\NormalTok{)}
\KeywordTok{install.packages}\NormalTok{(}\StringTok{"readxl"}\NormalTok{)}
\KeywordTok{install.packages}\NormalTok{(}\StringTok{"dplyr"}\NormalTok{)}
\end{Highlighting}
\end{Shaded}

Note in this last package that is a lower case letter L at the end. For
read EXCELL. It is not the number 1 (one).

⚠️ If you see a message such as the one below, check for a typo.

\begin{verbatim}
Warning in install.packages :
  package ‘ggplot’ is not available (for R version 3.5.1)
\end{verbatim}

The correct package name is \texttt{ggplot2}, not \texttt{ggplot}.


\end{document}
