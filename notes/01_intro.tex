\documentclass[]{article}
\usepackage{lmodern}
\usepackage{amssymb,amsmath}
\usepackage{ifxetex,ifluatex}
\usepackage{fixltx2e} % provides \textsubscript
\ifnum 0\ifxetex 1\fi\ifluatex 1\fi=0 % if pdftex
  \usepackage[T1]{fontenc}
  \usepackage[utf8]{inputenc}
\else % if luatex or xelatex
  \ifxetex
    \usepackage{mathspec}
  \else
    \usepackage{fontspec}
  \fi
  \defaultfontfeatures{Ligatures=TeX,Scale=MatchLowercase}
\fi
% use upquote if available, for straight quotes in verbatim environments
\IfFileExists{upquote.sty}{\usepackage{upquote}}{}
% use microtype if available
\IfFileExists{microtype.sty}{%
\usepackage{microtype}
\UseMicrotypeSet[protrusion]{basicmath} % disable protrusion for tt fonts
}{}
\usepackage[margin=1in]{geometry}
\usepackage{hyperref}
\hypersetup{unicode=true,
            pdftitle={Lesson 01 - Introduction to the class},
            pdfborder={0 0 0},
            breaklinks=true}
\urlstyle{same}  % don't use monospace font for urls
\usepackage{graphicx,grffile}
\makeatletter
\def\maxwidth{\ifdim\Gin@nat@width>\linewidth\linewidth\else\Gin@nat@width\fi}
\def\maxheight{\ifdim\Gin@nat@height>\textheight\textheight\else\Gin@nat@height\fi}
\makeatother
% Scale images if necessary, so that they will not overflow the page
% margins by default, and it is still possible to overwrite the defaults
% using explicit options in \includegraphics[width, height, ...]{}
\setkeys{Gin}{width=\maxwidth,height=\maxheight,keepaspectratio}
\IfFileExists{parskip.sty}{%
\usepackage{parskip}
}{% else
\setlength{\parindent}{0pt}
\setlength{\parskip}{6pt plus 2pt minus 1pt}
}
\setlength{\emergencystretch}{3em}  % prevent overfull lines
\providecommand{\tightlist}{%
  \setlength{\itemsep}{0pt}\setlength{\parskip}{0pt}}
\setcounter{secnumdepth}{0}
% Redefines (sub)paragraphs to behave more like sections
\ifx\paragraph\undefined\else
\let\oldparagraph\paragraph
\renewcommand{\paragraph}[1]{\oldparagraph{#1}\mbox{}}
\fi
\ifx\subparagraph\undefined\else
\let\oldsubparagraph\subparagraph
\renewcommand{\subparagraph}[1]{\oldsubparagraph{#1}\mbox{}}
\fi

%%% Use protect on footnotes to avoid problems with footnotes in titles
\let\rmarkdownfootnote\footnote%
\def\footnote{\protect\rmarkdownfootnote}

%%% Change title format to be more compact
\usepackage{titling}

% Create subtitle command for use in maketitle
\providecommand{\subtitle}[1]{
  \posttitle{
    \begin{center}\large#1\end{center}
    }
}

\setlength{\droptitle}{-2em}

  \title{Lesson 01 - Introduction to the class}
    \pretitle{\vspace{\droptitle}\centering\huge}
  \posttitle{\par}
    \author{}
    \preauthor{}\postauthor{}
      \predate{\centering\large\emph}
  \postdate{\par}
    \date{Last Updated 06-14-2019}


\begin{document}
\maketitle

\hypertarget{course-overview}{%
\section{Course Overview}\label{course-overview}}

This course is designed as a primer to get the complete novice up and
running with the basic knowledge of how to use the statistical
programming language R in an environment that emphasizes reproducible
research and literate programming for data analysis.

The target audience is anyone who wants to do their own data analysis.
The course will cumulate with an peer-evaluated exploratory data
analysis on either a pre-specified data set or your data set of choice.

\hypertarget{schedule-of-topics}{%
\subsection{Schedule of Topics}\label{schedule-of-topics}}

\begin{itemize}
\tightlist
\item
  Intro to the R language and it's packages
\item
  R Studio suite of awesomeness
\item
  Reproducible Research with R Markdown
\item
  Functions
\item
  Getting data into R
\item
  Managing the spreadsheet-like data frame
\item
  Data Visualization
\item
  Data Manipulation and Aggregation
\item
  Exploratory Data Analysis
\end{itemize}

\hypertarget{logistics}{%
\subsection{Logistics}\label{logistics}}

\begin{itemize}
\tightlist
\item
  All content is fully online on the course website:
  \url{https://norcalbiostat.github.io/MATH130/}
\item
  Class time is spent expanding on ideas and concepts, working through
  assignments as a class or in pairs.
\end{itemize}

\hypertarget{grading}{%
\subsection{Grading}\label{grading}}

\begin{itemize}
\tightlist
\item
  Credit / No Credit. There are 100 points available in this course
  through attendance, assignments, and a final project.
\item
  You must earn 70 points to receive credit for the course.
\item
  The \href{../syllabus.html}{syllabus} provides specific details on
  points per assignment.
\end{itemize}

\hypertarget{time-committment}{%
\subsection{Time Committment}\label{time-committment}}

\begin{itemize}
\tightlist
\item
  Fast paced, we're hitting the ground running and only have 5 weeks.
\item
  Not designed to teach you everything about R, just enough to make you
  dangerous.
\item
  Daily practice will pay off.
\item
  But can be very rewarding
\item
  Garbage in = garbage out
\item
  Effort in = satisfaction
\end{itemize}

\emph{We've all been there. It will get easier}

\hypertarget{tool-choices}{%
\section{Tool choices}\label{tool-choices}}

\hypertarget{why-use-r}{%
\subsection{Why use R?}\label{why-use-r}}

\begin{itemize}
\tightlist
\item
  Open source, and free
\item
  Rapid improvement
\item
  Tons of learning resources
\item
  Currently R is used in Chico's upper division Applied Statistics and
  Data Science courses.
\item
  Very popular in \href{https://r4ds.had.co.nz/}{Data Science},
  \href{https://www.core-econ.org/why-doing-economics-has-embraced-r/}{Economics},
  \href{https://www.nature.com/news/programming-tools-adventures-with-r-1.16609}{Natural
  Sciences}, \href{https://personality-project.org/r/}{Psychology} and
  many other fields.
\end{itemize}

\hypertarget{why-use-r-studio}{%
\subsection{Why use R Studio?}\label{why-use-r-studio}}

\begin{itemize}
\tightlist
\item
  Customizable workspace that docks all windows together.
\item
  Notebook formats that allow for easy sharing of code and output, and
  integration with other languages (Python, C++, SQL, Stan)
\item
  Syntax highlighting, warning errors when missing a closing
  parentheses.
\item
  Cross-platform interface. Also works on Windows/iOS/Linux.
\item
  Tab completion for functions. Forget the syntax or a variable name?
  Popup helpers are available.
\item
  Free training videos available from the developers directly.
\item
  Built in version control
\item
  One button publishing of reproducible documents such as reports,
  interactive visualizations, presentations (like this one!), websites.
\end{itemize}

\hypertarget{programming-is-scary}{%
\subsection{Programming is scary!}\label{programming-is-scary}}

Learning to program has other benefits

\begin{itemize}
\tightlist
\item
  Improves your logical skills and critical problem solving
\item
  Increases your attention to detail
\item
  Increases your self reliance and empowers you to control your own
  research.
\item
  Your PI will love your awesome graphics and reports.
\item
  Some people think what you do is magic.
\item
  Thinking graduate school?
  \href{http://www.nature.com/nature/journal/v541/n7638/full/nj7638-563a.html}{{[}expect
  to learn this on your own{]}}
\item
  \href{https://skillcrush.com/2015/01/28/laurence-bradford-10-reasons/}{{[}A
  few{]}}
  \href{https://careerfoundry.com/en/blog/web-development/7-benefits-of-learning-to-code/}{{[}other
  lists{]}}
  \href{https://skillcrush.com/2017/01/30/learn-to-code-benefits/}{{[}of
  reasons{]}}
\end{itemize}

\emph{No joke, this is the key to learning the details of how things
work}

\hypertarget{why-no-point-and-click}{%
\subsection{Why no point and click?}\label{why-no-point-and-click}}

Because it's not reproducible.

\begin{itemize}
\tightlist
\item
  Which boxes did you click last time?
\item
  New data? Gotta do it all over.
\item
  Need to expand your model? Gotta do it all over.
\item
  Made a mistake in the data coding? Gotta do it all over\ldots{}
\end{itemize}

\hypertarget{getting-help}{%
\section{Getting Help}\label{getting-help}}

\hypertarget{in-person}{%
\subsection{In Person}\label{in-person}}

\begin{itemize}
\tightlist
\item
  Data Science Initiative

  \begin{itemize}
  \tightlist
  \item
    \url{http://datascience.csuchico.edu} What the Data Science
    community at Chico is up to.
  \item
    Community Coding.

    \begin{itemize}
    \tightlist
    \item
      Dedicated space to work on coding projects.
    \item
      Collaborate with your peers and learn from experts.
    \item
      Need 1 unit? Enroll in Math 290-02 and attend at least 10 times.
    \item
      You DON'T need to be enrolled to attend!
    \end{itemize}
  \item
    Weekly seminars and workshops in a variety of languages and topics.
  \end{itemize}
\item
  The \href{https://www.meetup.com/Chico-R-Users-Group/}{R Users Meetup
  group} useful if you want to stay connected to the community and learn
  about upcoming events.

  \begin{itemize}
  \tightlist
  \item
    \url{https://www.meetup.com/Chico-R-Users-Group/}
  \end{itemize}
\end{itemize}

\emph{Sometimes a second pair of eyeballs is all you need}

\hypertarget{online}{%
\subsection{Online}\label{online}}

\begin{itemize}
\tightlist
\item
  In RStudio go to \texttt{Help} --\textgreater{} \texttt{Cheatsheets}
\item
  \href{https://groups.google.com/forum/\#!forum/chico-rug}{Chico R
  Users Google Group}. This is our discussion forum to ask (and answer)
  questions outside of class.

  \begin{itemize}
  \tightlist
  \item
    \url{https://groups.google.com/forum/\#!forum/chico-rug}
  \end{itemize}
\item
  Googling error messages:
  \url{https://www.google.com/search?q=common+R+error+messages}

  \begin{itemize}
  \tightlist
  \item
    Good blog by expert R contributor
    \url{http://varianceexplained.org/courses/errors/}
  \end{itemize}
\item
  Stack Overflow tag {[}r{]}
  \url{https://stackoverflow.com/questions/tagged/r}
\item
  R Studio community \url{https://community.rstudio.com/}
\item
  Twitter: \#rstats @Rstudio
\end{itemize}

\hypertarget{written}{%
\subsection{Written}\label{written}}

If you're a book kinda person, there is plenty of help available as
well. Many have online versions or free PDF's.

\begin{itemize}
\tightlist
\item
  R Markdown, the Definitive Guide:
  \url{https://bookdown.org/yihui/rmarkdown/}
\item
  R for Data Science \url{https://r4ds.had.co.nz/}
\item
  Cookbook for R \url{http://www.cookbook-r.com/}
\item
  R Graphics Cookbook (I use this all the time) -- Chapter 8 in the
  above link
\item
  The Art of R Programming \url{https://nostarch.com/artofr.htm}
\item
  R for\ldots{} \url{http://r4stats.com/}

  \begin{itemize}
  \tightlist
  \item
    Excel Users \url{https://www.rforexcelusers.com/}
  \item
    SAS and SPSS Users \url{http://r4stats.com/books/r4sas-spss/}
  \item
    STATA Users \url{http://r4stats.com/books/r4stata/}
  \end{itemize}
\end{itemize}


\end{document}
