\documentclass[]{article}
\usepackage{lmodern}
\usepackage{amssymb,amsmath}
\usepackage{ifxetex,ifluatex}
\usepackage{fixltx2e} % provides \textsubscript
\ifnum 0\ifxetex 1\fi\ifluatex 1\fi=0 % if pdftex
  \usepackage[T1]{fontenc}
  \usepackage[utf8]{inputenc}
\else % if luatex or xelatex
  \ifxetex
    \usepackage{mathspec}
  \else
    \usepackage{fontspec}
  \fi
  \defaultfontfeatures{Ligatures=TeX,Scale=MatchLowercase}
\fi
% use upquote if available, for straight quotes in verbatim environments
\IfFileExists{upquote.sty}{\usepackage{upquote}}{}
% use microtype if available
\IfFileExists{microtype.sty}{%
\usepackage{microtype}
\UseMicrotypeSet[protrusion]{basicmath} % disable protrusion for tt fonts
}{}
\usepackage[margin=1in]{geometry}
\usepackage{hyperref}
\hypersetup{unicode=true,
            pdftitle={Lesson 04 - Intro to R Markdown},
            pdfborder={0 0 0},
            breaklinks=true}
\urlstyle{same}  % don't use monospace font for urls
\usepackage{color}
\usepackage{fancyvrb}
\newcommand{\VerbBar}{|}
\newcommand{\VERB}{\Verb[commandchars=\\\{\}]}
\DefineVerbatimEnvironment{Highlighting}{Verbatim}{commandchars=\\\{\}}
% Add ',fontsize=\small' for more characters per line
\usepackage{framed}
\definecolor{shadecolor}{RGB}{248,248,248}
\newenvironment{Shaded}{\begin{snugshade}}{\end{snugshade}}
\newcommand{\AlertTok}[1]{\textcolor[rgb]{0.94,0.16,0.16}{#1}}
\newcommand{\AnnotationTok}[1]{\textcolor[rgb]{0.56,0.35,0.01}{\textbf{\textit{#1}}}}
\newcommand{\AttributeTok}[1]{\textcolor[rgb]{0.77,0.63,0.00}{#1}}
\newcommand{\BaseNTok}[1]{\textcolor[rgb]{0.00,0.00,0.81}{#1}}
\newcommand{\BuiltInTok}[1]{#1}
\newcommand{\CharTok}[1]{\textcolor[rgb]{0.31,0.60,0.02}{#1}}
\newcommand{\CommentTok}[1]{\textcolor[rgb]{0.56,0.35,0.01}{\textit{#1}}}
\newcommand{\CommentVarTok}[1]{\textcolor[rgb]{0.56,0.35,0.01}{\textbf{\textit{#1}}}}
\newcommand{\ConstantTok}[1]{\textcolor[rgb]{0.00,0.00,0.00}{#1}}
\newcommand{\ControlFlowTok}[1]{\textcolor[rgb]{0.13,0.29,0.53}{\textbf{#1}}}
\newcommand{\DataTypeTok}[1]{\textcolor[rgb]{0.13,0.29,0.53}{#1}}
\newcommand{\DecValTok}[1]{\textcolor[rgb]{0.00,0.00,0.81}{#1}}
\newcommand{\DocumentationTok}[1]{\textcolor[rgb]{0.56,0.35,0.01}{\textbf{\textit{#1}}}}
\newcommand{\ErrorTok}[1]{\textcolor[rgb]{0.64,0.00,0.00}{\textbf{#1}}}
\newcommand{\ExtensionTok}[1]{#1}
\newcommand{\FloatTok}[1]{\textcolor[rgb]{0.00,0.00,0.81}{#1}}
\newcommand{\FunctionTok}[1]{\textcolor[rgb]{0.00,0.00,0.00}{#1}}
\newcommand{\ImportTok}[1]{#1}
\newcommand{\InformationTok}[1]{\textcolor[rgb]{0.56,0.35,0.01}{\textbf{\textit{#1}}}}
\newcommand{\KeywordTok}[1]{\textcolor[rgb]{0.13,0.29,0.53}{\textbf{#1}}}
\newcommand{\NormalTok}[1]{#1}
\newcommand{\OperatorTok}[1]{\textcolor[rgb]{0.81,0.36,0.00}{\textbf{#1}}}
\newcommand{\OtherTok}[1]{\textcolor[rgb]{0.56,0.35,0.01}{#1}}
\newcommand{\PreprocessorTok}[1]{\textcolor[rgb]{0.56,0.35,0.01}{\textit{#1}}}
\newcommand{\RegionMarkerTok}[1]{#1}
\newcommand{\SpecialCharTok}[1]{\textcolor[rgb]{0.00,0.00,0.00}{#1}}
\newcommand{\SpecialStringTok}[1]{\textcolor[rgb]{0.31,0.60,0.02}{#1}}
\newcommand{\StringTok}[1]{\textcolor[rgb]{0.31,0.60,0.02}{#1}}
\newcommand{\VariableTok}[1]{\textcolor[rgb]{0.00,0.00,0.00}{#1}}
\newcommand{\VerbatimStringTok}[1]{\textcolor[rgb]{0.31,0.60,0.02}{#1}}
\newcommand{\WarningTok}[1]{\textcolor[rgb]{0.56,0.35,0.01}{\textbf{\textit{#1}}}}
\usepackage{graphicx,grffile}
\makeatletter
\def\maxwidth{\ifdim\Gin@nat@width>\linewidth\linewidth\else\Gin@nat@width\fi}
\def\maxheight{\ifdim\Gin@nat@height>\textheight\textheight\else\Gin@nat@height\fi}
\makeatother
% Scale images if necessary, so that they will not overflow the page
% margins by default, and it is still possible to overwrite the defaults
% using explicit options in \includegraphics[width, height, ...]{}
\setkeys{Gin}{width=\maxwidth,height=\maxheight,keepaspectratio}
\IfFileExists{parskip.sty}{%
\usepackage{parskip}
}{% else
\setlength{\parindent}{0pt}
\setlength{\parskip}{6pt plus 2pt minus 1pt}
}
\setlength{\emergencystretch}{3em}  % prevent overfull lines
\providecommand{\tightlist}{%
  \setlength{\itemsep}{0pt}\setlength{\parskip}{0pt}}
\setcounter{secnumdepth}{0}
% Redefines (sub)paragraphs to behave more like sections
\ifx\paragraph\undefined\else
\let\oldparagraph\paragraph
\renewcommand{\paragraph}[1]{\oldparagraph{#1}\mbox{}}
\fi
\ifx\subparagraph\undefined\else
\let\oldsubparagraph\subparagraph
\renewcommand{\subparagraph}[1]{\oldsubparagraph{#1}\mbox{}}
\fi

%%% Use protect on footnotes to avoid problems with footnotes in titles
\let\rmarkdownfootnote\footnote%
\def\footnote{\protect\rmarkdownfootnote}

%%% Change title format to be more compact
\usepackage{titling}

% Create subtitle command for use in maketitle
\providecommand{\subtitle}[1]{
  \posttitle{
    \begin{center}\large#1\end{center}
    }
}

\setlength{\droptitle}{-2em}

  \title{Lesson 04 - Intro to R Markdown}
    \pretitle{\vspace{\droptitle}\centering\huge}
  \posttitle{\par}
    \author{}
    \preauthor{}\postauthor{}
      \predate{\centering\large\emph}
  \postdate{\par}
    \date{Last Updated 06-14-2019}


\begin{document}
\maketitle

\hypertarget{introduction}{%
\section{Introduction}\label{introduction}}

In this lesson you will learn to write a document using R markdown,
integrate live R code into a literate statistical program, compile R
markdown documents using \texttt{knitr} and related tools, and organize
a data analysis sandbox so that it is reproducible and accessible to
others.

\hypertarget{student-learning-objectives}{%
\subsubsection{Student Learning
Objectives}\label{student-learning-objectives}}

After completing this lesson students will be able to create a
reproducible R Markdown document that integrates written text, R code
and output into a literate document.

\hypertarget{requirements}{%
\section{Requirements}\label{requirements}}

\begin{itemize}
\tightlist
\item
  R and R Studio must be installed
\item
  The following R packages must be installed: \texttt{rmarkdown} and
  \texttt{knitr}. (See Lesson 02 for instructions)
\end{itemize}

\hypertarget{appearance}{%
\section{Appearance}\label{appearance}}

R Markdown combines normal text like this, code (shown in a colored box
below) and the output from the code directly following with two
proceeding pound signs (\#).

\begin{Shaded}
\begin{Highlighting}[]
\DecValTok{2}\OperatorTok{+}\DecValTok{2}
\end{Highlighting}
\end{Shaded}

\begin{verbatim}
## [1] 4
\end{verbatim}

Code chunks start with three back ticks (to the left of the 1) and an r
in brackets \texttt{}\{r\}`. They close (end) with another three back
ticks. Note the background color has changed to a ligher shade. This
helps you identify you have closed your code chunk properly.

You can insert code chunks by using the button in the top right of an
RMD file (Insert --\textgreater{} R), or by typing CTRL + ALT + I.

Only code goes in code chunks. That's why they're called \textbf{code}
chunks. No normal text. All explanatory text goes outside a code chunk.

R Markdown files have a file extension of \texttt{.Rmd}.

\hypertarget{test-your-setup}{%
\section{Test your setup}\label{test-your-setup}}

Let's create your first markdown file!

\begin{enumerate}
\def\labelenumi{\arabic{enumi}.}
\tightlist
\item
  In R Studio go to \emph{File} --\textgreater{} \emph{New File}
  --\textgreater{} \emph{R Markdown}
\item
  Title this document \textbf{My First R Markdown Document}, then click
  OK.
\item
  Click the small blue disk icon to save this file into your class
  folder.
\item
  Save this file using the file name \textbf{test\_markdown\_document}.

  \begin{itemize}
  \tightlist
  \item
    \textless{}U+26A0\textgreater{}\textless{}U+FE0F\textgreater{} File
    names cannot have spaces or special characters.
  \item
    \textless{}U+26A0\textgreater{}\textless{}U+FE0F\textgreater{}Do not
    specify the file type. It will be set automatically.
  \end{itemize}
\item
  Click the \textbf{KNIT} button (has the yarn ball next to it) to
  convert this file into HTML.
\item
  Look at the HTML file that was created. You should be able to match
  the code with the resulting output.
\end{enumerate}

This is what we mean by reproducible. If you make a change in the code
document, and re-knit (aka compile), your changes will be reflected in
the generated document.

\begin{quote}
Try this:\\
1. Change the code from \texttt{summary(cars)} to
\texttt{summary(iris)}.\\
2. Write a sentence below
\end{quote}

\hypertarget{start-the-first-homework}{%
\section{Start the first homework}\label{start-the-first-homework}}

\begin{enumerate}
\def\labelenumi{\arabic{enumi}.}
\tightlist
\item
  \textbf{Right click} and select \textbf{save as} (or save target as)
  to download \href{../hw/HW1.Rmd}{{[}HW 1.Rmd{]}} code file to your
  class folder.
\item
  Navigate to your class folder and double click to open this file in R
  Studio

  \begin{itemize}
  \tightlist
  \item
    You might have to tell your computer what program to use.
  \item
    \textless{}U+274C\textgreater{}\textless{}U+274C\textgreater{} Do
    NOT open this file from your browser window.
    \textless{}U+274C\textgreater{}\textless{}U+274C\textgreater{}
  \end{itemize}
\item
  Knit this file to HTML.
\item
  Look at each piece of the output around problem 0. Mentally match each
  piece of output with the corresponding section in the RMD file.
\item
  This provides a homework template for you to use to write your
  assignment. Write your answers directly into this document.
\item
  Submit the \textbf{RMD} file before the due date. I will knit the file
  on my machine and grade the result. Make sure it looks good before
  turning it in!
\end{enumerate}


\end{document}
